\documentclass{llncs}
%%%%%%%%%%%%%%%%%%%%%%%%%%%%%%%%%%%%%%%%%%%%%%%%%%%%%%%%%%%
%% package sillabazione italiana e uso lettere accentate
\usepackage[utf8]{inputenc}
\usepackage[english]{babel}
\usepackage[T1]{fontenc}
\usepackage[final]{listings}
\usepackage{color}
\usepackage{caption}
%%%%%%%%%%%%%%%%%%%%%%%%%%%%%%%%%%%%%%%%%%%%%%%%%%%%%%%%%%%%%

\usepackage{url}
\usepackage{xspace}

\makeatletter
%%%%%%%%%%%%%%%%%%%%%%%%%%%%%% User specified LaTeX commands.
\usepackage{manifest}

\makeatother


%%%%%%%
 \newif\ifpdf
 \ifx\pdfoutput\undefined
 \pdffalse % we are not running PDFLaTeX
 \else
 \pdfoutput=1 % we are running PDFLaTeX
 \pdftrue
 \fi
%%%%%%%
 \ifpdf
 \usepackage[pdftex]{graphicx}
 \else
 \usepackage{graphicx}
 \fi
%%%%%%%%%%%%%%%
 \ifpdf
 \DeclareGraphicsExtensions{.pdf, .jpg, .tif}
 \else
 \DeclareGraphicsExtensions{.eps, .jpg}
 \fi
%%%%%%%%%%%%%%%

\graphicspath{{img/}}
\captionsetup[lstlisting]{position=bottom}

\newcommand{\java}{\textsf{Java}}
\newcommand{\contact}{\emph{Contact}}
\newcommand{\corecl}{\texttt{corecl}}
\newcommand{\medcl}{\texttt{medcl}}
\newcommand{\msgcl}{\texttt{msgcl}}
\newcommand{\android}{\texttt{Android}}
\newcommand{\dsl}{\texttt{DSL}}
\newcommand{\jazz}{\texttt{Jazz}}
\newcommand{\rtc}{\texttt{RTC}}
\newcommand{\ide}{\texttt{Contact-ide}}
\newcommand{\xtext}{\texttt{XText}}
\newcommand{\xpand}{\texttt{Xpand}}
\newcommand{\xtend}{\texttt{Xtend}}
\newcommand{\pojo}{\texttt{POJO}}
\newcommand{\junit}{\texttt{JUnit}}

\newcommand{\action}[1]{\texttt{#1}\xspace}
\newcommand{\code}[1]{{\small{\texttt{#1}}}\xspace}
\newcommand{\codescript}[1]{{\scriptsize{\texttt{#1}}}\xspace}

% Cross-referencing
\newcommand{\labelsec}[1]{\label{sec:#1}}
\newcommand{\xs}[1]{\sectionname~\ref{sec:#1}}
\newcommand{\xsp}[1]{\sectionname~\ref{sec:#1} \onpagename~\pageref{sec:#1}}
\newcommand{\labelssec}[1]{\label{ssec:#1}}
\newcommand{\xss}[1]{\subsectionname~\ref{ssec:#1}}
\newcommand{\xssp}[1]{\subsectionname~\ref{ssec:#1} \onpagename~\pageref{ssec:#1}}
\newcommand{\labelsssec}[1]{\label{sssec:#1}}
\newcommand{\xsss}[1]{\subsectionname~\ref{sssec:#1}}
\newcommand{\xsssp}[1]{\subsectionname~\ref{sssec:#1} \onpagename~\pageref{sssec:#1}}
\newcommand{\labelfig}[1]{\label{fig:#1}}
\newcommand{\xf}[1]{\figurename~\ref{fig:#1}}
\newcommand{\xfp}[1]{\figurename~\ref{fig:#1} \onpagename~\pageref{fig:#1}}
\newcommand{\labeltab}[1]{\label{tab:#1}}
\newcommand{\xt}[1]{\tablename~\ref{tab:#1}}
\newcommand{\xtp}[1]{\tablename~\ref{tab:#1} \onpagename~\pageref{tab:#1}}
% Category Names
\newcommand{\sectionname}{Section}
\newcommand{\subsectionname}{Subsection}
\newcommand{\sectionsname}{Sections}
\newcommand{\subsectionsname}{Subsections}
\newcommand{\secname}{\sectionname}
\newcommand{\ssecname}{\subsectionname}
\newcommand{\secsname}{\sectionsname}
\newcommand{\ssecsname}{\subsectionsname}
\newcommand{\onpagename}{on page}

\newcommand{\xauthA}{Alessandro Bagnoli }
\newcommand{\xauthB}{Filippo Nicolini }
\newcommand{\xauthC}{Luca Pascucci }
\newcommand{\xfaculty}{II Faculty of Engineering}
\newcommand{\xunibo}{Alma Mater Studiorum -- University of Bologna}
\newcommand{\xaddrBO}{viale Risorgimento 2}
\newcommand{\xaddrCE}{via Sacchi 3}
\newcommand{\xcityBO}{40136 Bologna, Italy}
\newcommand{\xcityCE}{47023 Cesena, Italy}

%
% Comments
%
%%% \newcommand{\todo}[1]{\bf{TODO:}\emph{#1}}

\definecolor{keywords}{rgb}{0.5,0,0.35}
\definecolor{greencomments}{rgb}{0,0.5,0}
\definecolor{redstrings}{rgb}{0.9,0,0}

\definecolor{dkgreen}{rgb}{0.4,0.4,0.4}
\definecolor{gray}{rgb}{0.5,0.5,0.5}
\definecolor{dkgray}{rgb}{0.2,0.2,0.2}
\definecolor{commentgray}{rgb}{0.6,0.6,0.6}
\definecolor{mauve}{rgb}{0.58,0,0.82}

% Define Language
\lstdefinelanguage{qa}
{
	% list of keywords
	morekeywords={
		RobotSystem,
		QActor,
		Event,
		Dispatch,
		Context,
		context,
		EventHandler,
		ip,
		host,
		port,
		-httpserver,
		for,
		Plan,
		normal,
		actions,
		println,
		actorOp,
		switchTo,
		transition,
		whenTime,
		whenEvent,
		onEvent,
		reactive,
		whenEnd,
		whenMsg,
		whenTout,
		or,
		sound,
		time,
		file,
		onMsg,
		Rules,
		Robot,
		robotForward,
		speed,
		robotStop,
		robotLeft,
		robotRight
	},
	sensitive=true, % keywords are not case-sensitive
	morecomment=[l]{//}, % l is for line comment
	morecomment=[s]{/*}{*/}, % s is for start and end delimiter
	morestring=[b]" % defines that strings are enclosed in double quotes
}

\lstdefinestyle{qa}{language=qa,
	frame=single,
	captionpos=b,
	aboveskip=3mm,
	belowskip=3mm,
	showstringspaces=false,
	columns=flexible,
	basicstyle={\fontsize{10}{10}\ttfamily},
	numbers=left,
	numberstyle=\tiny\color{gray},
	keywordstyle=\color{keywords},
	commentstyle=\color{greencomments},
	stringstyle=\color{blue},
	breaklines=true,
	breakatwhitespace=true,
	tabsize=3,
	frame=single,
	rulecolor=\color{black}
}

\lstdefinestyle{java}{language=Java,
	frame=single,
	captionpos=b,
	aboveskip=3mm,
	belowskip=3mm,
	showstringspaces=false,
	columns=flexible,
	basicstyle={\fontsize{10}{10}\ttfamily},
	numbers=left,
	numberstyle=\tiny\color{gray},
	keywordstyle=\color{keywords},
	commentstyle=\color{greencomments},
	stringstyle=\color{blue},
	breaklines=true,
	breakatwhitespace=true,
	tabsize=3,
	frame=single,
	rulecolor=\color{black}
}

\begin{document}

\title{Final Task 2016 \\ Software Engineering
 process report}

\author{\xauthA, \xauthB \and \xauthC}
%%\author{\xauthA}

\institute{%
  \xunibo\\\xaddrCE, \xcityCE\\\email{\{alessandro.bagnoli4, filippo.nicolini2, luca.pascucci\}@studio.unibo.it}
%%%  \xunibo\\\xaddrCE, \xcityCE\\\email\ nameA.studentA@studio.unibo.it
}

\maketitle

%% \begin{abstract}
%% \footnotesize
%%This a Latex template to be used for the reports of Software Engineering.
%%\keywords{Software engineering, managed software development, reports, ....}
%%\end{abstract}

%%% \sloppy

\section{Introduction}
\labelsec{intro}
This report describe the process for develop a system composed by a robot, one or multiple sonar and a radar. It includes all the different steps needed to analyse, prototype and implement the system, with a better definition of itself in the "Goals" and "Requirements" sections.
\section{Vision}
\labelsec{Vision}
A list of notions learned in this course:
\begin{itemize}
	\item A design without specifications cannot be right or wrong, it can only be surprising!
	\item There is non code without a project, no project without problem analysis and no problem without requirements.
	\item The question is how to make them explicit, effective and reusable.
	\item A feature does not exist unless a test validates that it functions.
	\item Analyse a little. Design a little. Code a little. Test what you can.
\end{itemize}
Our vision resumes part of the above-mentioned notions, understand QA language, be able to define a logical architecture and create a model of the system that should be originated from the problem and not from code already available because a software component can be conceived as a logical (abstract) entity that can be implemented in several ways.
\section{Goals}
\labelsec{Goals}
The goal is to build a software system able to evolve from an initial proptotype (defined as the result of a problem analysis phase) to a final, testable product, by working in a team and by "mixing" in a proper (pragmatically useful) way agile (SCRUM) software development with modelling.
\section{Requirements}
\labelsec{Requirements}
A differential drive robot (called from now on robot) must reach an area (B) starting from a given point A. To reach the area B, the robot must cross an area equipped with N (N>=1) distance sensors (sonars). The signal emitted by each sonar is reflected by a wall put in front of it at a distance of approximately 90 cm.
\begin{figure}[h]
	\centering
	\includegraphics[width=9.5cm]{system.png}
\end{figure}
\\
Design and build a (prototype of a) software system such that:
\begin{itemize}
	\item shows the sonar data on the GUI associated to a console running on a conventional PC
	\item evaluates the expression: \\
			$ (s_{k} + s_{k+1} + ... s_{N}) / (N-k+1) $ \\
			where k is the number of the first sensor not reached by the robot and sk is the value of the distance currently measured by that sensor. If the value of the expression is less than a prefixed value DMIN( e.g. DMIN=70), play an alarm sound.
	\item when the robot reaches the area in front of a sonar, it
			\begin{enumerate}
				\item first stops
				\item then rotates to its left of approximately 90 degrees
				\item starts blinking a led put on the robot
				\item takes a photo of the wall (in a simulated way only, if no WebCam is available) and sends the photo to console by using the MQTT protocol
				\item rotates to its right of approximately 90 degrees to compensate the previous rotation
				\item stops the blinking of the led and continues its movement towards the area B
			\end{enumerate}
	\item when the robot leaves the area in front to the last sonar, it continues until it arrives at the area B
	\item stops the robot movement as soon as possible:
	\begin{itemize}
		\item when an obstacle is detected by the sonar in front of the robot
		\item when an alarm sound is played
		\item the user sends to the robot a proper command (e.g. STOP)
	\end{itemize}
	\item makes it possible to restart the system (by manually repositioning the robot at point A) without restarting the software
\end{itemize}

\section{Requirement analysis}
\labelsec{ReqAnalysis}

\subsection{User stories}
\labelssec{UserStories}
\textbf{As a} user, \textbf{I want} to place the robot at the beginning point, and make it start running. \\
\textbf{As a} user, \textbf{I can} stop the robot at any time, and make it restart placing it at the beginning point. \\
\textbf{As a} user, \textbf{I want} to see the photos taken by the robot during its walking. \\
\textbf{As a} user, \textbf{I want} to see the sonars data on a graphical user interface.
\subsection{Scenarios}
\labelssec{Scenarios}
A scenario describes the ways in which the user interacts with the system. Furthermore, the scenarios can be used to gather stories or to build requirements.
The scenarios of the system described are:
\begin{itemize}
\item The user needs to place in the correct way all the parts of the system.
\item The user stops the robot pushing the button on the console.
\item The user can visualize the photo sent by the robot.
\item The user can check on the graphical interface the status of the sonars.
\end{itemize}
\subsection{(Domain)model}
A domain model is a conceptual model that incorporate behaviour and data. It can be used to solve problems related to that domain.
\subsection{Test plan}
The test plan include integration test that is used to check that the integration between the components takes place in the right way.
\section{Problem analysis}
\labelsec{ProblemAnalysis}
Our technological hypothesis is Java and Object Oriented Programming. We
will show that we need to develop a new infrastructure because the only features
offered by the programming language chosen are not enough. To keep our code
well developed, scalable, efficient and maintainable, we will use the most known
development patterns. We must define an interaction language, that needs a
communication standard.
\subsection{Logic architecture}
//TODO inserire ddr
\subsection{Abstraction gap}
During the problem analysis we considered Java as our technological hypothesis.
We realized that Java’s unique communication tool is procedure calls, with
OO paradigm it becomes really difficult to implement modular communication
where entities are completely independent and loosely coupled. We need a new
way to let our components interact. In agreement with our vision, we understand
that the best way to overcome this problem is to develop a new software
infrastructure that will map more strictly to our model representation than Java
paradigms. This infrastructure will be used not only in this project, but every
future developed process that will have the same needs because, according to
our visions, we want to develop reusable code.
Our platform must offer some functionalities. First of all we need a formal definition
of a System:
a System is one or more active entities that interact to reach some goal or provide
some functionalities. A system is composed by one or more Contexts, a Contexts
identifies a logical location at deployment time. The System enables the communication
between the entities and synchronizes them so Entities must interact
using the System only. In this way a System can be concentrated or distributed
seamlessly for the application designer. Defined this general perspective we can
introduce two paradigms:
\begin{itemize}
	\item Message Based Programming
	\item Event Based Programming
\end{itemize}
\textbf{Message Based Programming:} to introduce message based programming we
need to define the entities that can exchange messages. We will call these entities
QActors.
QActors in our platform are not message-driven but message-based. The core
difference between message driven and message based is that message driven
systems can be only reactive to messages instead of being able to decide when
messages and which messages are functional to evaluate. In this way we can
model not only reactive entities but also proactive ones. QActors live in a context
and they can communicate with other QActors using different communication
primitives, like \textit{dispatch}: an asynchronous message without returning information. \\
QActors can receive messages using the receive message primitives, like \textit{receiveMessage} which extracts the first message from the actor message queue. \\\\
\textbf{Event Based Programming}: we needed to define the concept of events as pervasive messages that are spread in the system. The message passing paradigm
lacks of this feature because all messages are point to point. Events are a form
of asynchronous communication that is not point to point. In our system our
entities can declare to be interested in an event, when that event will be emitted
the entities can react in some way.
When we introduced proactive QActors we realized that we needed the concept
of Plan. A Plan is a sequence of actions (you can see actions as programming
language instructions), every plan has its own logic and can switch to a different
plan. When the execution of a plan reaches its end it can specify if the
previous plan must continue its execution or suspend it. This abstraction was
lacking of the better part of message/event driven programming: reactivity. This
gap is filled by Asynchronous Actions, which are tipically time
consuming actions that can be executed in blocking or non-blocking way. When
runing an asynchronous action its execution can be interrupted by specified
events and the QActor must react executing the associated plan. The logic of
plan switching is defined as above. QActors are able to execute actions in two
ways:
\begin{itemize}
	\item executing compiled code
	\item interpreting meta code on the fly
\end{itemize} 
Adding the possibility to interpret code on the fly permits to the QActor to enrich
its behaviour during runtime. The just described platform will be realized with
Java Programming Language (our technologic assumption) but will be made
to be interoperable with other technologies by the extensive use of text-based
messages using the standard socket technology.
On top of the API offered by the platform we’ll realize a DSL using the xtext
framework. With the use of such tool we’ll generate a declarative language that
will provide two main benefits:
\begin{itemize}
	\item a formal and human readable language to describe the problem analysis,
	entities interactions and system topology;
	\item code generation to be able to customize and execute the artifacts generated by the DSL interpretation.
\end{itemize}
With such tools in mind we can reimagine the traditional problem analysis
techniques and tackle the problems in a formal and more convenient way.
\subsection{Risk analysis}
The only remarkable risk in this development process is that the platform we will realize will be technology dependent to Java Runtime Environment. The DSL is completely technology independent but the code generation feature will be bounded to the platform’s API written in Java Language. If the deployment environment can’t take advantage of the Java Platform our development process will be slower because we can’t make use of the developed platform.
\section{Work plan}
\labelsec{wplan}
The point is that we have to specify a model, so we must carefully understand the detail level of the specification.
Our work plan is set as follows:
\begin{enumerate}
	\item find the main subsystems and define the system contexts;
	\item define the structure of the events that can occur in the system;
	\item define the structure of the messages exchanged by the actors;
	\item define the main actors working in each context;
	\item define the type of the logical interaction among the actors;
	\item define the logical behaviour of each actor according to interaction constraints;
	\item build and test the new system components (unit testing);
	\item build and test the system (integration testing).
\end{enumerate}
After the problem analysis, a document called product backlog was defined, which defines the main issues to be solved to obtain a complete software product according to the requirements of the customer. Each issue was associated with a score from 1 to 5 that represents the cost to realize it  estimated by the members of the development team.\\
\begin{minipage}{\linewidth}
\centering
\captionof{table}{\textbf{Product Backlog}} \label{tab:title} 
\end{minipage}
\begin{tabular}{| c | p{9.5cm} | c |}
\hline
\textbf{Priority} & \textbf{Issue} &  \textbf{Score}\\ \hline
1 & Robot goes forward and stops when its sonar notice an obstacle & 2 \\  \hline
2 & Sonar reads and emits distance values to the radar & 1 \\  \hline
3 & Radar displays sonar's values on GUI & 2 \\  \hline
4 & Create user console and stop robot when user command is send & 2 \\ \hline
5 & Robot stops when reach a sonar and starts its routine & 2 \\  \hline
6 & Handle and blink robot Led &  1 \\ \hline
7 & Robot captures photo and sends it and console saves it locally & 4 \\ \hline
8 & Radar evaluates expression and plays alarm sound & 3 \\ \hline
9 & User restart system without restarting software & 2 \\
\hline
\end{tabular}
\\\\\\
Product Backlog contains a limited number of items because the infrastructure used to realize the system allows to solve most of the issues discussed.
\section{Project}
\labelsec{Project}
The purpose of the project phase is to refine the logical architecture of the system, considering all the binding aspects that have been ignored in the previous phases.
\subsection{Structure}
Distributed systems are composed of a set of active entities (QActor) each working in a (different) computational environemnt (Context). In this case scenario we have three different Contexts, each of which represents a different computational node:
\begin{itemize}
	\item \textbf{ctxRadar}: represents the computational node "PC";
	\item \textbf{ctxSensorEmitter}: represents the computational node "Sonar";
	\item \textbf{ctxRobot}: represents the computational node "Robot".
\end{itemize}
\subsection{Interaction}
A QActor can interact with others (local or remote) QActor by sending/receiving messages. A QActor can also emit or sense events.
\begin{figure}[h]
	\centering
	\includegraphics[width=\linewidth]{interaction.png}
	\caption{Interactions between the three contexts.}
\end{figure}
\subsection{Behaviour}
A QActor is an active (autonomous) entity that runs in parallel with the other actors defined in the same or in other contexts. The behaviour of a QActor can be expressed as a Moore Finite State Machine (M-FSM). Each state of the M-FSM can be expressed by a \textbf{Plan} which is composed of a sequence of \textbf{actions}.
Here are the finite state machines of our system:
\begin{figure}[h]
	\centering
	\includegraphics[width=\linewidth]{radargui.png}
	\caption{FSM for the radargui actor.}
\end{figure}
\begin{figure}[h]
	\centering
	\includegraphics[width=\linewidth]{alarmhandler.png}
	\caption{FSM for the alarmhandler actor.}
\end{figure}
\begin{figure}[h]
	\centering
	\includegraphics[width=\linewidth]{photoreceiver.png}
	\caption{FSM for the photoreceiver actor.}
\end{figure}
\begin{figure}[h]
	\centering
	\includegraphics[width=\linewidth]{sensorsonar.png}
	\caption{FSM for the sensorsonar actor.}
\end{figure}
\begin{figure}[h]
	\centering
	\includegraphics[width=\linewidth]{robotactor.png}
	\caption{FSM for the robot actor.}
\end{figure}
\clearpage
\section{Implementation}
\labelsec{Implementation}
\subsection{Radar}
The console part of the radar is implemented as a web page, which is automatically generated in the \lstinline[columns=fixed, style=java]{srcMore} directory in a package associated with each Context when the \textbf{-httpserver} flag for a Context is set. It is implemented by a HTTP web-socket server working on port 8080. This interface emits different kind of events. For example, the Start button emits the event  \lstinline[columns=fixed, style=java]{cmd : cmd(start)}.
\begin{figure}[h]
	\centering
	\includegraphics[width=\linewidth]{webconsole.png}
	\caption{Web console.}
\end{figure}
\\
The GUI part of the radar (where the sonars data are shown) is implemented using specific libraries provided by the software house.
\subsection{Led blinking}
The blink of the led has to be asynchronous, so that while the led is blinking, the robot can continue its operations without being blocked. In order to do so, the led is implemented as a \lstinline[columns=fixed, style=java]{SituatedActiveObject} supplied by the software house.
\lstinputlisting[style=java]{list/BlinkAsynch.java}
\subsection{Transmission of images through MQTT}
For the photo shoot through the camera, we used an open source library found on Github, called \textbf{JRPiCam} which wraps the commands needed to take a photo. Once the photo is acquired, it is converted in a String encoded with \textbf{Base64} encoding and then published on the \textbf{MQTT} broker, with a specified topic. The radar running on PC will have to decode the String received through MQTT in order to rebuild the image and then save it locally.
\lstinputlisting[style=java]{list/Camera.java}
\lstinputlisting[style=java]{list/Photoreceiver.java}
All the sonars, either on the path or on the robot, retreive the distances executing a C program called \lstinline[columns=fixed, style=java]{SonarAlone}, provided by the sofware house.
\lstinputlisting[style=java]{list/Sensorsonar.java}
The software house has already studied, in previous works, how to handle a robot-related device that by its nature can not send messages / events. When you say that an actor is a Robot, the code generator produces two files in the package reserved for the actor: a prolog theory \lstinline[columns=fixed, style=java]{sensorTheory} and a java file called \lstinline[columns=fixed, style=java]{SensorObserver}. The observer was automatically created and registered with all the sensors supplied to the robot so that they can perceive state changes: it is a task of the application designer to specify the behaviour after receiving a sensory data. To do so, you need to modify the generated prolog theory or specify the behaviour in the java file. \\
Despite this, we decided to handle the sonar on the robot the same way we handle the sonars on the path, as the data returned are far more accurate.
\section{Testing}
\labelsec{testing}
The system was initially tested using the \textit{robotMock} abstraction locally. The robot mock is totally equivalent to the ddr robot from the point of view of the behaviours, but uses simulated sensory data and generates prints instead of sending electric signals to the pins connected to the motors. The system has successfully passed the testing phase.
\section{Deployment}
\labelsec{Deployment}
The robot actor runs on a Raspberry Pi on the physical robot, while the radar actor runs on a conventional PC. Each sensorsonar actor runs on a different Raspberry Pi. The system can start only when robot, radar, and at least one sensorsonar actor are running.
\section{Maintenance}
\labelsec{Maintenance}
The system is easily expandable and reusable. The configuration of the system is extremely simple with the introduction of Prolog rules and Java methods containing the variable system's parameters. The choices made guarantee also strong flexibility even with new or modified requirements: for example, adding new sensors do not involve drastic changes to the system.
\newpage
%===========================================================================
\section{Information about the authors}
\labelsec{Author}

\vskip.5cm
%%% \begin{figure}
\begin{tabular}{ | c |  }
	\hline
	% after \\: \hline or \cline{col1-col2} \cline{col3-col4} ...
	Alessandro Bagnoli
	\\
	\hline
	\includegraphics[width = 5cm]{img/FilippoNicolini.jpg}
	\\
	\hline
\end{tabular}
\vskip0cm
\noindent
\begin{tabular}{ | c |  }
	\hline
	% after \\: \hline or \cline{col1-col2} \cline{col3-col4} ...
	Filippo Nicolini 
	\\
	\hline
	\includegraphics[width = 5cm]{img/FilippoNicolini.jpg}
	\\
	\hline
\end{tabular}
\vskip0cm
\noindent
\begin{tabular}{ | c |  }
	\hline
	% after \\: \hline or \cline{col1-col2} \cline{col3-col4} ...
	Luca Pascucci 
	\\
	\hline
	\includegraphics[width = 5cm]{img/LucaPascucci.jpg}
	\\
	\hline
\end{tabular}


%===========================================================================


%%% \begin{itemize}
%%% \item Titolo di studio:\\ \\
%%% \item Interessi particolari:\\ \\
%%% \item Ha sostenuto fino ad oggi il seguente numero di esami:\\ \\
%%% \item Deve ancora sostenere i seguenti esami del I anno:\\ \\
%%% \item Prevede di svolgere un tirocinio presso:\\ \\
%%% \item Prevede di laurearsi nella sessione:\\ \\
%%% \item Intende proseguire gli studi per conseguire: \\  \\  \\
%%%   	presso la sede universitaria di: \\ \\
%%% \item Intende entrare subito nel mondo del lavoro presso : \\ \\
%%% \end{itemize}

\end{document}
