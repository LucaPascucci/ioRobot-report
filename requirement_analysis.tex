\section{Requirement analysis}
\labelsec{ReqAnalysis}
One way to set up the analysis is to pay particular attention to the nouns and verbs that appear in the requirements. From a systematic analysis of nouns it is possible to formalize knowledge on the application domain and favor the production of a domain model based on abstraction sets (domain entity) characterized by common property, policies and constraints. From the analysis of verbs it is possible to identify the inevitable actions that the system will have to perform.
\begin{minipage}{\linewidth}
\centering
\captionof{table}{\textbf{Glossary}} \label{tab:title} 
\end{minipage}
\begin{tabular}{| l | p{10cm} |}
\hline
\textbf{Area A} & Robot starting point \\ \hline
\textbf{Robot} & A differential drive robot is a mobile robot whose movement is based on two separately driven wheels placed on either side of the robot body. It can thus change its direction by varying the relative rate of rotation of its wheels and hence does not require an additional steering motion. The robot is equipped with devices like a distance sensor (sonar) and (optionally) with a Web Cam both positioned in its front. It owns also a Led. The robot should move from A Area to B Area by travelling along a straight line, at a distance of approximately 40-50 cm from the base-line of the sonars. \\ \hline
\textbf{Sonar} & A distance sensor is a sensor able to detect the presence of nearby objects without any physical contact. \\ \hline
\textbf{Led} & A led is an output device that can show its internal state emitting or not emitting a coloured light (i.e. green or red Led). \\ \hline
\textbf{Blink} & Turning on and off a led with regular intervals. \\ \hline
\textbf{Console} & An interface where an operator can send commands to the robot. It's a remote interface that run on a different device like a PC.\\ \hline
\textbf{User Command} & An order given to the robot by the operator through the Console. \\ \hline
\textbf{Radar} & Graphic User Interface that will shows the sonar data to the user \\ \hline
\textbf{Alarm sound} & Alarm played when the expression's value is less than a prefix value. A notice will be sent by the system to the robot to stop it\\ \hline
\textbf{Area B} & Robot finish point \\ \hline
\end{tabular}

\subsection{User stories}
\labelssec{UserStories}
%\textbf{As a} user, \textbf{I want} to place the robot at the beginning point, and make it start running. \\
%\textbf{As a} user, \textbf{I can} stop the robot at any time, and make it restart placing it at the beginning point. \\
%\textbf{As a} user, \textbf{I want} to see the photos taken by the robot during its walking. \\
%\textbf{As a} user, \textbf{I want} to see the sonars data on a graphical user interface.
%\\\\
\textbf{As a} user, \textbf{I want} (R1) to place the robot at the beginning point, and make it start running to a prefixed area. While the robot is moving and reaches the area in front of a sonar, \textbf{I want} that (R2) the robot stops turn left about 90$^\circ$, start blinking a led, take the photo of the world in front of it, send the photo to the console, turn right about 90$^\circ$ to compensate the previous rotation, stop blinking the led and continues its movement to the prefixed area. While the robot is moving \textbf{I want} (R3) to be able to stop it, and make it restart placing it at the beginning point. \textbf{As a} user \textbf{I want} (R4) to see the sonar data on a graphical user interface associated to a console running on a conventional PC. \textbf{As a} user \textbf{I want} that (R5) the system evaluate the expression and play an alarm sound when its value is less than a prefixed value, stopping the robot. \textbf{As a} user \textbf{I want} that (R6) the robot can stop itself when an obstacle is detected by the sonar in front of it.
\subsection{(Domain)model}
A domain model is a conceptual model that incorporate behaviour and data. It can be used to solve problems related to that domain.
\subsection{Test plan}
The test plan include integration test that is used to check that the integration between the components takes place in the right way.