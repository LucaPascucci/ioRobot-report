\section{Requirement analysis}
\labelsec{ReqAnalysis}
The requirement analysis is an activity done before the development of a software system and the purpose of this analysis is to define the functionalities of the system. This is a basic phase for all software life cycle models, because in this stage the analyst must write all the specifics to describe the whole system. \\\\
\textbf{Glossary}\\\\
\begin{tabular}{| l | p{11cm} |}
\hline
\textbf{Robot} & A differential drive robot is a mobile robot whose movement is based on two separately driven wheels placed on either side of the robot body. It can thus change its direction by varying the relative rate of rotation of its wheels and hence does not require an additional steering motion. The robot is equipped with a distance sensor (sonar) and (optionally) with a Web Cam both positioned in its front. It owns also a Led. The robot should move from A to B by travelling along a straight line, at a distance of approximately 40-50 cm from the base-line of the sonars. \\ \hline
\textbf{Sonar} & A distance sensor is a sensor able to detect the presence of nearby objects without any physical contact. \\ \hline
\textbf{Led} & A led is an output device that can show its internal state emitting or not emitting a coloured light (i.e. green or red Led). \\ \hline
\textbf{Blink} & Turning on and off a led with regular intervals. \\ \hline
\textbf{User Interface} & An interface where an operator can send commands to the robot. \\ \hline
\textbf{User Comand} & An order given to the robot by the operator through the User Interface. \\
\hline
\end{tabular}

\subsection{User stories}
\labelssec{UserStories}
\textbf{As a} user, \textbf{I want} to place the robot at the beginning point, and make it start running. \\
\textbf{As a} user, \textbf{I can} stop the robot at any time, and make it restart placing it at the beginning point. \\
\textbf{As a} user, \textbf{I want} to see the photos taken by the robot during its walking. \\
\textbf{As a} user, \textbf{I want} to see the sonars data on a graphical user interface.
\subsection{(Domain)model}
A domain model is a conceptual model that incorporate behaviour and data. It can be used to solve problems related to that domain.
\subsection{Test plan}
The test plan include integration test that is used to check that the integration between the components takes place in the right way.