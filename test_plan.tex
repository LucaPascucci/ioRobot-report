\subsection{Test plan}
One of the biggest advantage in using the language / metamodel \textit{qa} is the possibility to obtain an executable program before the implementation phase. The chance to show the customer a working prototype of the system after analyzing the requirements reduces the cost: if this was not possible, a mistake due to improper interpretation of one of the requirements could be identified in the advanced stages of the software product design path (e.g. at the end of the implementation phase), and all the the work previously done could even be invalidated.
\\In order to test the proper functioning of individual components (\textit{Unit testing}), the models realized during the requirements analysis were connected to mock components (components capable of simulating the behaviour of a specific entity). This way, it's possible to determine whether the behaviour is the expected one or not. However, because the actors interact with each other during the tests, these tests are also \textit{integration tests}.