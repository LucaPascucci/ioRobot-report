\section{Vision}
\labelsec{Vision}
Here we’re collecting some visions that will inspire the development of this project and software in general:
\begin{itemize}
	\item A design without specifications cannot be right or wrong, it can only be surprising!
	\item There is no code without a project, no project without problem analysis and no problem without requirements. The question is how to make them explicit, effective and reusable;
	\item A feature does not exist unless a test validates that it works;
	\item Analyse a little. Design a little. Code a little. Test what you can;
	\item The team that develops the tests should be different from the one who will realize the project;
	\item Zooming? Trying to get closer... We should look at things initially from a far point of view like a black box and then trying to get closer to the details of the system, white box.
	\item We should always check the existence of prior projects, trends or patterns regarding the technological domain we are facing. If does exist something we should study it, not only to take advantage of it during the development but also to recognize its limits and to purpose some innovative approach coming from different technological domains;
	\item We should use a top down approach during the development phase and bottom-up approach during the implementation phase;
	\item The development of the project should be technology independent;
	\item We should find or create a formal language able to describe the results of the problem analysis. It should be similar to spoken language. This language could take advantage of different programming paradigms (functional, object oriented and so on) and different programming models (actor model, message passing model and so on). This language should not be tied to any technology implementation. There should be one or more parsers/compilers for this language that will generate source code for a specific platform. This generated code should be used as the skeleton of the real product.
\end{itemize}